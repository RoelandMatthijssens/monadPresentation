\begin{frame}
    \frametitle{Slightly contrived example}
    \begin{block}{Something you may want to do}
        \begin{itemize}
            \item Add debugging statements to function calls
            \item Remember: no destructive changes (adding to a global log list)
        \end{itemize}
    \end{block}
\end{frame}

\begin{frame}[fragile]
    \frametitle{Intuitive implementation}
    \begin{block}{}
        \begin{itemize}
            \item $f \: x \rightarrow (x+1, ``f \: was \: called")$
            \item $g \: x \rightarrow (x+2, ``g \: was \: called")$
        \end{itemize}
    \end{block}
    \begin{block}{}
        $f$ and $g$ both take a $float$ and return a tuple $(float, string)$
    \end{block}
\end{frame}

\begin{frame}[fragile]
    \frametitle{}
    \begin{block}{How to use this in composition $f \cdot g$}
        \begin{itemize}
            \item $y, logString_g = g(x)$
            \item $z, logString_f = f(y)$
            \item $log = logString_g + logString_f$
            \item $return (z, log)$
        \end{itemize}
\begin{lstlisting}
\end{lstlisting}
    \end{block}
    \begin{block}{}
        It can be done, but it is ugly and long winded
    \end{block}
\end{frame}

\begin{frame}[fragile]
    \frametitle{}
    \begin{block}{}
        \begin{itemize}
            \item if $f$ and $g$ simply took a $float$ and returned a $float$ we could write $f \cdot g$
            \item $f \: x \rightarrow x+1$
            \item $g \: x \rightarrow x+2$
            \item $f \cdot g \: 3 \rightarrow 6$
        \end{itemize}
    \end{block}
    \begin{block}{}
        \begin{itemize}
            \item But they take a $float$ and output a $(float, string)$. Simply composing them would not work
            \item $f \: x \rightarrow (x+1, ``f \: called.")$
            \item $g \: x \rightarrow (x+2, ``g \: called.")$
            \item $f \cdot g \: 3 \rightarrow$ Error: f takes only one argument but got 2
        \end{itemize}
    \end{block}
\end{frame}

\begin{frame}[fragile]
    \frametitle{}
    \begin{block}{}
        \begin{itemize}
            \item Lets redefine $f$ and $g$
            \item $f' \: x \: logSoFar \rightarrow (x+1, logSoFar + ``f \: called.")$
            \item $g' \: x \: logSoFar \rightarrow (x+2, logSoFar + ``g \: called.")$
            \item $f' \cdot g' \: 3 \: ``" \rightarrow (6, ``g \: called.f \: called.")$
            \item So now this works. Jeej
            \item Unfortunatly we don't have a $f'$ but an $f$ which is inconvenient
        \end{itemize}
    \end{block}
\end{frame}

\begin{frame}[fragile]
    \frametitle{Small Tangent}
    \begin{block}{Simple type definition}
        \begin{itemize}
            \item $f$ is a function that takes a $float$ and returns a $(float, string)$
            \item $f :: float \rightarrow (float, string)$
        \end{itemize}
    \end{block}
    \begin{block}{More complex type definition}
        \begin{itemize}
            \item $f$ is a function that takes a function $g$ and a $float$ and returns a $string$
            \item the function $g$ that is an argument of $f$ will take a $float$ and return a $(float, string)$
            \item $f :: g \rightarrow float \rightarrow string$
            \item $f :: (float \rightarrow (float, string)) \rightarrow float \rightarrow string$
        \end{itemize}
    \end{block}
\end{frame}

\begin{frame}[fragile]
    \frametitle{Bind}
    \begin{block}{}
        The main problem is that $f$ does not take $(float, string)$ but just $float$. Lets create $bind$, a higher
        order function that can help us
    \end{block}
    \begin{block}{Typedef of $bind$}
        \begin{itemize}
            \item Iinputs: bind takes a function that takes a $float$ and returns a $string$ and a $(float, string)$
            \item Output: a new function that will take a $(float, string)$ and returns a $(float, string)$
            \item $bind :: (float \rightarrow (float, string)) \rightarrow (float, string) \rightarrow ((float, string) \rightarrow (float, string))$
        \end{itemize}
    \end{block}
\end{frame}

\begin{frame}[fragile]
    \frametitle{}
    \begin{block}{Define $bind$}
        \begin{itemize}
            \item $bind \: f' \: (g_f, g_s) = $
            \item $\:\:\: let (f_f, f_s) = f'(g_f)$
            \item $\:\:\: in (f_f, g_s + f_s)$
        \end{itemize}
    \end{block}
\end{frame}

\begin{frame}[fragile]
    \frametitle{}
    \begin{block}{Cool, but what good does it do?}
        \begin{itemize}
            \item Given a pair of debuggable functions $f$ and $g$ of type $float \rightarrow (float, string)$
            \item We can easily compose them using $(bind \: f) \cdot g$
            \item Or simply $bind \: f \cdot g$
            \item Lets call this composition function $\bigotimes$
        \end{itemize}
    \end{block}
\end{frame}

\begin{frame}[fragile]
    \frametitle{}
    \begin{block}{}
        \begin{itemize}
            \item We now have half of our formal definition.
            \item Bear with me a bit longer
        \end{itemize}
    \end{block}
    \begin{block}{}
        \begin{itemize}
            \item Lets ask ourselves if we can create an identity (aka $unit$) function for this $bind$ method we just defined
            \item So that $unit \bigotimes f = f \bigotimes unit = f$
            \item Of course there is
            \item We can just define unit as $unit \: x \rightarrow (x, ``")$
        \end{itemize}
    \end{block}
\end{frame}

\begin{frame}[fragile]
    \frametitle{}
    \begin{block}{}
        \begin{itemize}
            \item The fun thing about this $unit$ function is that we can use it to $lift$ a normal function to a debuggable function
            \item $lift \: f = unit \cdot f$
            \item This means that I can now compose a normal function with a debuggable functions.
            \item $debugable \bigotimes lift \: normalfunction$
            \item Of cource the debug output of our lifted normal function is just the empty string (thats how we defined $unit$
        \end{itemize}
    \end{block}
\end{frame}

\begin{frame}[fragile]{}
    \begin{block}{Summary of example}
        \begin{itemize}
            \item We stared with simple functions
            \item We wanted to make them debuggable by having them output an additional string
            \item We need to be able to compose small functions into smaller ones
            \item We did not want to write a lot of boiler plate
            \item We accomplished this by implementing $bind$, $unit$, $lift$ and $\bigotimes$
        \end{itemize}
    \end{block}
\end{frame}
