\begin{frame}
    \frametitle{Slightly contrived example}
    \begin{block}{Something you may want to do}
        \begin{itemize}
            \item Add debugging statements to function calls
            \item Remember: no destructive changes (adding to a global log list)
        \end{itemize}
    \end{block}
\end{frame}

\begin{frame}[fragile]
    \frametitle{Intuitive implementation}
    \begin{block}{}
        \begin{minted}{haskell}
    f x -> (x + 1, 'f was called')
    g x -> (x + 1, 'g was called')
\end{minted}

    \end{block}
    \begin{block}{}
        $f$ and $g$ both take a $float$ and return a tuple $(float, string)$
    \end{block}
\end{frame}

\begin{frame}[fragile]
    \frametitle{}
    \begin{block}{How to use this in composition $f \cdot g$}
        \begin{minted}{haskell}
    y, logString_g = g(x)
    z, logString_f = f(y)
    log = logString_g + logString_f
    return (z, log)
\end{minted}

    \end{block}
    \begin{block}{}
        It can be done, but it is ugly and long winded
    \end{block}
\end{frame}

\begin{frame}[fragile]
    \frametitle{}
    \begin{block}{}
        if $f$ and $g$ simply took a $float$ and returned a $float$ we could write $f \cdot g$
        \begin{minted}{haskell}
f x -> x+1
g x -> x+2
f . g 3 -> 6
\end{minted}

    \end{block}
    \begin{block}{}
        But they take a $float$ and output a $(float, string)$. Simply composing them would not work
        \begin{minted}{haskell}
f x -> (x+1, "f called.")
g x -> (x+2, "g called.")
f . g 3 ->
 Error: f expects a float but received a tuple
\end{minted}

    \end{block}
\end{frame}

\begin{frame}[fragile]
    \frametitle{}
    \begin{block}{}
        Lets redefine $f$ and $g$
        \begin{minted}{haskell}
f' x log -> (x+1, log + "f called.")
g' x log -> (x+2, log + "g called.")
f' . g' 3 "" -> (6, "g called.f called.")
\end{minted}

        So now this works. Jeej \\
        Unfortunatly we don't have an $f'$ but an $f$ which is inconvenient
    \end{block}
\end{frame}

\begin{frame}[fragile]
    \frametitle{Small Tangent}
    \begin{block}{Simple type definition}
        $f$ is a function that takes 2 $int$s and returns an $int$
        \begin{minted}{haskell}
f :: float -> (float, string)
\end{minted}

    \end{block}
    \begin{block}{More complex type definition}
        $f$ is a function that takes a function $g$ and an $int$ and returns an $int$ \\
        the function $g$ that is an argument of $f$ will also take an $int$ and return an $int$
        \begin{minted}{haskell}
f :: g -> Int -> Int
f :: (Int->Int)->Int->Int
\end{minted}

        AKA currying
    \end{block}
\end{frame}

\begin{frame}[fragile]
    \frametitle{Bind}
    \begin{block}{}
        The main problem is that $f$ does not take $(float, string)$ but just $float$. Lets create $bind$, a higher
        order function that can help us
    \end{block}
    \begin{block}{Typedef of $bind$}
        \begin{itemize}
            \item Inputs: bind takes a function that takes a $float$ and returns a $(float,string)$ and a $(float, string)$
            \item Output: a new function that will take a $(float, string)$ and returns a $(float, string)$
        \end{itemize}
        \begin{minted}{haskell}
bind ::
    (float -> (float, string)) ->
    (float, string) ->
    ((float, string) -> (float, string))
\end{minted}

    \end{block}
\end{frame}

\begin{frame}[fragile]
    \frametitle{}
    \begin{block}{Define $bind$}
        \begin{minted}{haskell}
bind f (float, string) =
    let (new_float, new_string) = f(float)
    in (new_float, string + new_string)
\end{minted}

    \end{block}
\end{frame}

\begin{frame}[fragile]
    \frametitle{}
    \begin{block}{Cool, but what good does it do?}
        \begin{itemize}
            \item Given a pair of debuggable functions $f$ and $g$ of type $float \rightarrow (float, string)$
            \item We can easily compose them using $(bind \: f) \cdot g$
            \item Or simply $bind \: f \cdot g$
            \item Lets call this composition function $>>=$
            \item $f >>= g >>= h \: 3$
        \end{itemize}
    \end{block}
\end{frame}

\begin{frame}[fragile]
    \frametitle{}
    \begin{block}{}
        \begin{itemize}
            \item We now have half of our formal definition.
            \item Bear with me a bit longer
        \end{itemize}
    \end{block}
    \begin{block}{}
        \begin{itemize}
            \item Lets ask ourselves if we can create an identity (aka $unit$) function for this $bind$ method we just defined
            \item So that $(unit >>= f) = (f >>= unit) = f$
            \item Of course there is
            \item We can just define unit as $unit \: x \rightarrow (x, ``")$
        \end{itemize}
    \end{block}
\end{frame}

\begin{frame}[fragile]
    \frametitle{}
    \begin{block}{}
        \begin{itemize}
            \item The fun thing about this $unit$ function is that we can use it to $lift$ a normal function to a debuggable function
            \item $lift \: f = unit \cdot f$
            \item This means that I can now compose a normal function with a debuggable functions.
            \item $debugable >>= lift \: normalfunction$
            \item Of cource the debug output of our lifted normal function is just the empty string (thats how we defined $unit$
        \end{itemize}
    \end{block}
\end{frame}

\begin{frame}[fragile]{}
    \begin{block}{Summary of example}
        \begin{itemize}
            \item We started with simple functions
            \item We wanted to make them debuggable by having them output an additional string
            \item We need to be able to compose small functions into larger ones
            \item We did not want to write a lot of boiler plate
            \item We accomplished this by implementing $bind$, $unit$, $lift$ and $>>=$
        \end{itemize}
    \end{block}
\end{frame}
