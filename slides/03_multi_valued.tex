\begin{frame}[fragile]{}
    \begin{block}{Second small example}
        Before we give the formal definition lets do another quick example
    \end{block}
\end{frame}

\begin{frame}[fragile]{Multiple return values}
    \begin{block}{Real numbers}
        \begin{itemize}
            \item $sqrt :: real \rightarrow real$
            \item $cbrt :: real \rightarrow real$
            \item $sixthroot = sqrt \cdot cbrt$
        \end{itemize}
    \end{block}
    \begin{block}{Complex numbers}
        \begin{itemize}
            \item $sqrt :: complex \rightarrow [complex]$
            \item $cbrt :: complex \rightarrow [complex]$
            \item $sixthroot = \: ???$
        \end{itemize}
    \end{block}
\end{frame}

\begin{frame}[fragile]{}
    \begin{block}{How to do $sqrt \cdot cbrt$}
        \begin{itemize}
            \item First we apply the $cbrt$
            \item For each result in the returned list we apply $sqrt$
            \item This returns a new list of length 3 where each element is a list of length 2
            \item To form a single list of all 6 roots we simply flatten the list
        \end{itemize}
    \end{block}
\end{frame}

\begin{frame}[fragile]{}
    \begin{block}{Sounds familiar}
        \begin{itemize}
            \item $bind \: f \: x = concat ( map \: f \: x)$
            \item $unit \: x = [x]$
            \item $f \bigotimes g = bind \: f \cdot g$
            \item $lift \: f = unit \cdot f$
        \end{itemize}
    \end{block}
\end{frame}
